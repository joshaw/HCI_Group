\section{Group 6: University of Birmingham Homepage}

\subsection{Synopsis}

This group argues that the University of Birmingham website is poorly designed
and as a result fails to attract potentials students. They set out to design a
new website interface that tackles these problems and provides a better
experience for these users.

\subsection{Strengths of the Work}

\begin{itemize}
	\item
\end{itemize}

\subsection{Weakness of the Work}

\begin{itemize}

	\item  In researching related work, they neglected to evaluate websites
	other than university websites. Furthermore, in their conclusion of this
	research they did not outline which features in particular they would
	consider taking forward into their own design.

	\item After opting to focus only on students, there was still little
	variation between their individual user personas. For example, each of the
	personas were technically proficient.

	\item Their first generation prototypes were largely very similar,
	particularly prototypes 2 and 3, apart from some variations in the
	positions of the links.  This could be a symptom of having researched only
	university websites.

	\item Again, in their conclusion they did not highlight which features they
	would take forward from the first prototypes to the second prototype.

	\item They did not explain how they would evaluate their first prototypes.
	In particular, they used heuristic evaluations (page 36) before explaining
	why they would be using them (page 66).

\end{itemize}

\subsection{Summary}
