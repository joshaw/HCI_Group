% !TEX TS-program = pdflatex
%DIF LATEXDIFF DIFFERENCE FILE
%DIF DEL OtherProjectsReview-oldtmp-20027.tex   Wed Mar 26 09:42:51 2014
%DIF ADD OtherProjectsReview.tex                Sat Mar 22 11:49:01 2014
% !TEX encoding = UTF-8 Unicode

\documentclass[11pt]{article} % use larger type; default would be 10pt

\usepackage[utf8]{inputenc} % set input encoding (not needed with XeLaTeX)

%%% PAGE DIMENSIONS
\usepackage[top=0.6in, left=0.8in, right=0.8in, bottom=0.7in]{geometry} % to change the page dimensions
\geometry{a4paper} % or letterpaper (US) or a5paper or....
% \geometry{margins=2in} % for example, change the margins to 2 inches all round
% \geometry{landscape} % set up the page for landscape

\usepackage{graphicx} % support the \includegraphics command and options

\usepackage[parfill]{parskip} % Activate to begin paragraphs with an empty line rather than an indent

%%% PACKAGES
\usepackage{booktabs} % for much better looking tables
\usepackage{array} % for better arrays (eg matrices) in maths
%\usepackage{paralist} % very flexible & customisable lists (eg. enumerate/itemize, etc.)
\usepackage{verbatim} % adds environment for commenting out blocks of text & for better verbatim
\usepackage{subfig} % make it possible to include more than one captioned figure/table in a single float
\usepackage{mathtools} % for math environments like align
\usepackage{amssymb} % for symbols like \therefore

%%% OPTIONAL PACKAGES
%\usepackage{braket}

%%% HEADERS & FOOTERS
\usepackage{fancyhdr} % This should be set AFTER setting up the page geometry
\pagestyle{fancy} % options: empty , plain , fancy
\renewcommand{\headrulewidth}{0pt} % customise the layout...
\lhead{}\chead{}\rhead{}
\lfoot{}\cfoot{\thepage}\rfoot{}

%%% SECTION TITLE APPEARANCE
%\usepackage{sectsty}
%\allsectionsfont{\sffamily\mdseries\upshape} % (See the fntguide.pdf for font help)

%%% ToC (table of contents) APPEARANCE
%\usepackage[nottoc,notlof,notlot]{tocbibind} % Put the bibliography in the ToC
%\usepackage[titles,subfigure]{tocloft} % Alter the style of the Table of Contents
%\renewcommand{\cftsecfont}{\rmfamily\mdseries\upshape}
%\renewcommand{\cftsecpagefont}{\rmfamily\mdseries\upshape} % No bold!

%%% END Article customizations

% \author{Josh Wainwright \\ UID:1079596}
\author{}

\title{Human Computer Interaction \\ Review of Project Groups 6 \& 7}
\author{Group 5}
\date{28th March 2014}
%DIF PREAMBLE EXTENSION ADDED BY LATEXDIFF
%DIF UNDERLINE PREAMBLE %DIF PREAMBLE
\RequirePackage[normalem]{ulem} %DIF PREAMBLE
\RequirePackage{color}\definecolor{RED}{rgb}{1,0,0}\definecolor{BLUE}{rgb}{0,0,1} %DIF PREAMBLE
\providecommand{\DIFadd}[1]{{\protect\color{blue}\uwave{#1}}} %DIF PREAMBLE
\providecommand{\DIFdel}[1]{{\protect\color{red}\sout{#1}}}                      %DIF PREAMBLE
%DIF SAFE PREAMBLE %DIF PREAMBLE
\providecommand{\DIFaddbegin}{} %DIF PREAMBLE
\providecommand{\DIFaddend}{} %DIF PREAMBLE
\providecommand{\DIFdelbegin}{} %DIF PREAMBLE
\providecommand{\DIFdelend}{} %DIF PREAMBLE
%DIF FLOATSAFE PREAMBLE %DIF PREAMBLE
\providecommand{\DIFaddFL}[1]{\DIFadd{#1}} %DIF PREAMBLE
\providecommand{\DIFdelFL}[1]{\DIFdel{#1}} %DIF PREAMBLE
\providecommand{\DIFaddbeginFL}{} %DIF PREAMBLE
\providecommand{\DIFaddendFL}{} %DIF PREAMBLE
\providecommand{\DIFdelbeginFL}{} %DIF PREAMBLE
\providecommand{\DIFdelendFL}{} %DIF PREAMBLE
%DIF END PREAMBLE EXTENSION ADDED BY LATEXDIFF

\begin{document}

\maketitle
\section{Group 6: University of Birmingham Homepage}

\subsection{Synopsis}

This group argues that the University of Birmingham website is poorly designed
and as a result fails to attract potentials students. They set out to design a
new website interface that tackles these problems and provides a better
experience for these users.

\subsection{Strengths of the Work}

\begin{itemize}

	\item  The group evaluated many examples of related work, i.e.\ university
	websites

	\item Their research also includes principles of web design, and how
	information should be displayed on a web page.

	\item The scenarios created covered a wide-range of needs.

	\item The group conducted thorough evaluations of first generation
	prototypes using user personas and scenarios

\end{itemize}

\subsection{Weakness of the Work}

\begin{itemize}

	\item  In researching related work, they neglected to evaluate websites
	other than university websites. Furthermore, in their conclusion of this
	research they did not outline which features in particular they would
	consider taking forward into their own design.

	\item After opting to focus only on students, there was still little
	variation between their individual user personas. For example, each of the
	personas were technically proficient.

	\item Their first generation prototypes were largely very similar,
	particularly prototypes 2 and 3, apart from some variations in the
	positions of the links.  This could be a symptom of having researched only
	university websites.

	\item Again, in their conclusion they did not highlight which features they
	would take forward from the first prototypes to the second prototype.

	\item They did not explain how they would evaluate their first prototypes.
	In particular, they used heuristic evaluations (page 36) before explaining
	why they would be using them (page 66).

\end{itemize}

\subsection{Summary}

The group set out to improve the university’s website by focussing on the needs
of students. Their research covered a number of other university websites and
principles of web design. However, the research and initial prototypes focussed
solely on the homepage, whereas the group attempted to redesign other pages in
the second prototype. This led to design inconsistencies between different
pages. Focussing on a few good examples of university websites (or other
websites) and researching them more thoroughly by looking at subpages and using
their findings in their prototypes may have helped them achieve their goal.

\section{Group 7: Clothes Shopping Online}

\subsection{Synopsis}

\subsection{Strengths of the Work}

\begin{itemize}
	\item
\end{itemize}

\subsection{Weakness of the Work}

\begin{itemize}
	\item
\end{itemize}

\subsection{Summary}


\end{document}
