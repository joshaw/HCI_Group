\section{Group 7: Clothes Shopping Online}

\subsection{Synopsis}

This group wants to create an online clothes shopping experience tailored to
women who are unsure of what size to choose and whether it will flatter their
particular figure. This system aims to make it easier to buy clothes that fit
well and give women confidence that the clothes they buy will look how they
imagined without having to try them on first.

\subsection{Strengths of the Work}

\begin{itemize}

	\item The initial research of existing websites was very thorough. A large
	number of different clothes shops with from a range of different target
	audiences were examined.

	\item A specific and detailed evaluation plan was used.

	\item Complete and detailed user profiles for a range of ages with clear
	scenarios which outline several interesting possible use cases.

	\item Thorough evaluation of first generation prototypes using user
	scenarios.

\end{itemize}

\subsection{Weakness of the Work}

\begin{itemize}

	\item There is no related research into design principles and general
	guidelines. Also no research into online shopping websites which are not
	directly clothes related.

	\item Same prototyping tool used for all prototypes, and not much variation
	between first generation prototypes. This group failed to take advantage of
	exploring different designs and using the strengths from each one to
	develop a more improved design

	\item There is no clear definition of the aim of the project in the project
	specification.

	\item The group did not use a system usability test to analyse their
	results after real users experimented with their system. Their results were
	also nowhere to be seen.

	\item User scenarios did not cover a wide range of use cases, eg product
	sold out, email notifications for back in stock, refunding an item, etc.

\end{itemize}

\subsection{Summary}

The research of existing products encompassed a large number of existing
systems and plans for the evaluation of the products to be designed in this
project were thorough and clearly well thought through with respect to the end
users of the system. However, since there were so many already existing
websites that provide this functionality, the final prototype did not seem to
offer much that was radically different, or that would make this system better
or easier to use than the competition.
