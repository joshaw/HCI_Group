\subsection{User Input}
\label{sub:user_input}

In order to present the user with useful information, our application will have
to accept information from them to be processed. This is done by means of
forms, text input and buttons. To maintain a clean, intuitive user interface, a
simplistic approach is often taken to reduce the thinking time required to
process information on a single screen. If more information is required,
multiple screens are often used.

\subsubsection{Timetables}
\label{ssub:timetables}

A common set of information presented to a user which represents a considerable
challenge, particularly on small screens, is a timetable of available or
appropriate times.  When too much information is displayed on a single screen,
this can become confusing or impossible to read. For example, in
figure~\ref{fig:GoogleCalendarTimetable}\cite{GoogleCalendar}, despite a single
hour being a common appointment length, the text for these slots is hidden
entirely.

\begin{figure}[ht]
    \centering
    \begin{subfigure}[b]{0.25\textwidth}
        \includegraphics[width=\textwidth]{img/GoogleCalendarTimetable.png}
        \caption{Google Calendar}\label{fig:GoogleCalendarTimetable}
    \end{subfigure}%
    \qquad
    \begin{subfigure}[b]{0.25\textwidth}
        \includegraphics[width=\textwidth]{img/RedSpottedHankyTicketSelection}
        \caption{RedSpottedHanky}\label{fig:RedSpottedHankyTicketSelection}
    \end{subfigure}
    \caption{When too much information is displayed on a screen, it can be
    hard to read and interpret, whereas condensing the information and
    splitting it so that only currently appropriate information is shown
    makes it much easier to understand.}\label{fig:timetables}
\end{figure}

A common way to improve the readability of theses complex structures, which
often contain a large quantity of data, is to have graded selection of that
data. In other words, where there is an option to refine a search to reduce the
data needed to be displayed, only display the immediately relevant information,
but with simple navigation to other relevant data.

This method can be seen clearly in the RedSpottedHanky.com application,
figure~\ref{fig:RedSpottedHankyTicketSelection} when a user is searching for
tickets for a specific date and time. Although there may be many trains within
a narrow time gap, the application shows a small number of tickets with the
option to move either earlier or later. Each ticket time is also associated
with a number of options relating to ticket price. These are shown only for the
currently selected ticket time.

% subsubsection timetables (end)

\subsubsection{Date/Time Selection}
\label{ssub:date_time_selection}

In order to reduce the search range, often a date and/or time selection
dialogue is used. Figure~\ref{fig:date_time_selection} shows two methods this
is achieved.

Figure~\ref{fig:RedSpottedHankyDateTime}, on the right is an example, again
from RedSpottedHanky\cite{RedSpottedHanky}, which shows the time selection
associated with booking a train ticket. This design fails since the method of
changing the time requires very close control if accuracy is required, and is
time consuming if the desired time is far from the currently selected time. The
movement is performed in single increments or decrements of the hours and
minutes. This is despite the functionality described above which lets the user
view and switch to other trains at nearby times.

Figures~\ref{fig:GoogleCalendarDateTime}, in the middle
and~\ref{fig:GoogleCalendarDateTime2} on the left are examples from Google
Calendar which shows how the process can be made much more intuitive, simple
and fast, through the use of separate screens with large and clear selection.
This selection is much easier to navigate than the scrolling method used.

\begin{figure}[ht]
    \centering
    \begin{subfigure}[b]{0.25\textwidth}
        \includegraphics[width=\textwidth]{img/RedSpottedHankyDateTime}
        \caption{RedSpottedHanky}\label{fig:RedSpottedHankyDateTime}
    \end{subfigure}%
    \qquad
    \begin{subfigure}[b]{0.25\textwidth}
        \includegraphics[width=\textwidth]{img/GoogleCalendarDateTime}
        \caption{Google Calendar}\label{fig:GoogleCalendarDateTime}
    \end{subfigure}
    \qquad
    \begin{subfigure}[b]{0.25\textwidth}
        \includegraphics[width=\textwidth]{img/GoogleCalendarDateTime2}
        \caption{Google Calendar}\label{fig:GoogleCalendarDateTime2}
    \end{subfigure}
    \caption{The time selection for RedSpottedHanky requires the user to
    spend to much time selecting the time when larger increments could be
    used to smoothen the process. Google Calendar, on the other hand,
    allows simple and fast selection of the hours and minutes through
    separate screens.}\label{fig:date_time_selection}
\end{figure}

A combination of both of these is used in the stock iOS, shown in
figure~\ref{fig:iOSDateTime}\cite{iOSDateTime} where a much easier to navigate
scrolling mechanism is used. Though this can still cause the user to spend more
time selecting the correct number, the fact that the used can ``flick scroll''
though the numbers means reaching a value that is far from the currently
selected one is much quicker than the RedSpottedHanky application.

\begin{figure}[ht]
    \centering
    \begin{subfigure}[b]{0.25\textwidth}
        \includegraphics[width=\textwidth]{img/iOSDateTime}
        \caption{RedSpottedHanky}
    \end{subfigure}%
    \qquad
    \begin{subfigure}[b]{0.35\textwidth}
        \includegraphics[width=\textwidth]{img/iOS7DateTime}
        \caption{Google Calendar}
    \end{subfigure}
    \caption{Stock iOS date and time picker is easier to scroll through,
    but still requires more time than selecting the appropriate
    number.}\label{fig:iOSDateTime}
\end{figure}

% subsubsection date_time_selection (end)

\subsubsection{Forms}
\label{ssub:forms}

When entering information that is not limited to a small set of possible
values, sunch as a name, location or arbitary number, a form must be used to
accept the user input. Since touch screens rely of the user being able to
navigate to to correct form section, the input must be of sufficient size to
allow this movement easily. Firgure~\ref{fig:TescoFormInput} show a simple form
with two input boxes. Each has a clear border around it so the the target for
interaction is easier to tap.

An important consideration that has been made here is to specify that, for the
second set of text input, the data is strictly limited to digits. For this
reason, the keyboard switches from a general purpose ``querty'' keyboard to a
purely numerical version. Again, this assists the user, both by indicating that
only the provided digits are acceptable, and making the input of those digits
easier (often the numbers on a touch screen keyboard are only accessible by
switching modes).

\begin{figure}[ht]
    \centering
    \begin{subfigure}[b]{0.25\textwidth}
        \includegraphics[width=\textwidth]{img/TescoFormInput}
        \caption{Large, easy to access text entry}\label{fig:TescoFormInput}
    \end{subfigure}%
    \qquad
    \begin{subfigure}[b]{0.3\textwidth}
        \includegraphics[width=\textwidth]{img/PaintCalculatorCrowded}
        \caption{Crowded interfaces make entry more difficult.}
    \end{subfigure}
    \caption{}\label{fig:PaintCalculatorCrowded}
\end{figure}

% subsubsection forms (end)

% subsection user_input (end)

% vim: ft=tex
