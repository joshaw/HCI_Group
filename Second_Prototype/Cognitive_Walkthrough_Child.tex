\subsection{Task 4}
\label{ssub:task_4}

\textbf{Book an outdoor sport, with a professional coach, close to home.}

\textbf{User:} Child
\begin{enumerate}
	\item Press `New Booking'.
	\item Press `Sport' on the search screen.
	\item Select the `Outdoors' selection option.
	\item Press `Location' on the search screen.
	\item Select a maximum distance using the distance slider.
	\item Ensure the `Public transport' option is selected and choose a
		preferred travel time.
	\item Press `Done' to return to the previous screen, and then `Search'.
	\item Select `Distance' in the `Sort by' options and `Sport' in `Group by'.
	\item Browse the list of sports offered at the locations found.
	\item Select a result and select weekly times to book.
	\item Press `Book Now'.
\end{enumerate}

\subsubsection{Walkthrough}

\begin{description}
	\item[Selecting sports to book] Joe does not know exactly which sport it is
		he would like to book, so he is hesitant about entering the `Sport'
		menu as he thinks he might be required to select one from a list. Once
		finding that there is a more relaxed selection menu, so that he can
		specify just outdoor sports, he is confident he will find something he
		likes.

	\item[Specifying distance and travel method] When choosing the distance he
		is prepared to travel, Joe is unsure about the distances he is used to
		going so doesn't know what value to give. He knows how long it takes to
		get to school, so enters a value slightly higher than this for the
		public transport slider and doesn't move the radius slider.

	\item[Sorting and grouping results] Joe has to wait for the results for a
		few seconds because the search criteria are quite broad. During this
		time, there is little notification that progress is being made. When
		the results are shown, he quickly sorts by the distance, as he isn't
		interested in the price (Pete will pay) or the time (during the summer
		holidays). Joe has to experiment a couple of times with the `Group by'
		options as he doesn't quite understand their purpose as he
		misunderstands it to mean a bulk purchase.

	\item[Making repeated bookings] Joe is pleased that he can book several
		sessions easily at once as he quickly recognises that the `Repeat
		Bookings' section shows the same sport at the same time, but on
		different days. However, after selecting the bookings for the next few
		weeks, he realises he has added one week too many to the list of
		bookings to be made. He is not sure that clicking `Book Now' will allow
		him to remove these mistaken bookings.
\end{description}

\renewcommand{\arraystretch}{1.3}
\begin{longtabu}{cccccl}
	\toprule
	\textbf{Step} & \textbf{Criteria 1} & \textbf{Criteria 2} &
	\textbf{Criteria 3} & \textbf{Criteria 4} & \textbf{Success?} \\
	\midrule
	1  & y & y & y & y & Success \\
	2  & y & n & y & y & Success \\
	3  & y & n & y & y & Success \\
	4  & y & y & y & y & Success \\
	5  & y & y & n & n & Success \\
	6  & y & y & y & y & Success \\
	7  & y & y & y & n & Success \\
	8  & y & n & n & y & Fail    \\
	9  & y & y & y & y & Success \\
	10 & y & n & n & y & Fail    \\
	11 & y & y & y & y & Success \\
	\bottomrule
\end{longtabu}
