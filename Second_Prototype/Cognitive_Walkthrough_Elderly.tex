\subsubsection{Task 1}
\label{ssub:task_1}

\textbf{Booking a nearby swimming lane with disabled access tomorrow at
midday.}

\textbf{User:} Elderly
\begin{enumerate}
	\item Press `New Booking'.
	\item Press `Sport' on the search screen.
	\item Ensure only `Swimming' is ticked in the list of sports and press `Done'.
	\item Press `Location' on the search screen.
	\item Ensure `Disabled Access' is ticked and press `Done'.
	\item Press `Date' on the search screen.
	\item Ensure only tomorrow's date is highlighted on the calendar and press `Done'.
	\item Press `Time' on the search screen and press `Done'.
	\item Ensure only `12 pm' is selected.
	\item Press `Search' on the search screen and press `Done'.
	\item Press the top result in the list.
	\item Press `Book Now'.
\end{enumerate}

\paragraph{Walkthrough}

\begin{description}
	\item[Choosing new booking] Howard knows he wants to make a new booking and
		sees this option clearly on the screen. He hesitates slightly after
		seeing `My Details' and wondered if this is where he needs to go to say
		he has disability requirements.

	\item[Selecting swimming] Howard sees the list of search options on the
		main search screen and notices that next to `Sport' it says `any'. He
		knows wants to swim only so he presses the arrow which shows him a list
		of sports. On this screen, he is not sure what he is supposed to do as
		he cannot see swimming. Eventually, he presses the middle of the screen
		and realises the list moves up and down. He scrolls until he finds
		swimming and presses it, noticing that a blue tick appears after
		pressing it. He sees the done button and is returned to the previous
		screen with `any' now replaced by a small icon depicting a person
		swimming.

	\item[Selecting disabled access] Howard is not sure where he could look for
		disabled facilities. On the location screen, he notices box labeled
		`Disabled Access' and presses it noticing that a tick appears.

	\item[Selecting date and time] Next to date and time in the search screen
		Howard sees `Today' and `any' respectively. He presses each of these in
		turn knowing that he wants to search for midday tomorrow. On the date
		screen he presses the date he wants on the calendar and sees that it
		changes colour. However, today's date is also the same colour and he
		tries to to press today's date to change it. The buttons are quite
		close to each other so it takes some time for Howard to correctly press
		the right one. He does the same thing on time screen for midday and
		notices that the main search screen says `Tomorrow' and `Midday'.

	\item[Completing the search and choosing the booking] The main search
		screen informs Howard that each criteria is as he wishes. At first
		Howard doesn't realise that the word `Search' at the bottom of the
		screen is a button he can press to see the results. Eventually, he
		presses it and sees a list of bookings with different times and
		locations and presses the first one on the screen. Howard is shown more
		details about this booking and is satisfied with and eventually sees
		`Book Now' on the screens and presses that button.
\end{description}

\renewcommand{\arraystretch}{1.3}
\begin{longtabu}{cccccl}
	\toprule
	\textbf{Step} & \textbf{Criteria 1} & \textbf{Criteria 2} & \textbf{Criteria 3} & \textbf{Criteria 4} & \textbf{Success?} \\
	\midrule
	1  & y & y & y & y & Success \\
	2  & y & y & y & y & Success \\
	3  & n & y & y & y & Success \\
	4  & n & n & y & y & Fail    \\
	5  & n & n & y & y & Fail    \\
	6  & y & y & y & y & Success \\
	7  & y & y & y & n & Success \\
	8  & y & y & y & y & Success \\
	9  & y & y & y & y & Success \\
	10 & y & n & y & y & Success \\
	11 & y & y & y & y & Success \\
	12 & y & y & y & y & Success \\
	\bottomrule
\end{longtabu}
