As with the first prototypes, the second prototype was assessed against
Neilsen’s heuristics. The application achieves a neutral or positive result
against most of the principles but needs improvement in the following areas:
\begin{itemize}
	\item error recognition, and correction,
	\item flexibility and efficiency of use.
\end{itemize}

The application was also tested against user personas and their scenarios. The
new prototype combines elements of the previous designs, which means it is
possible for the user to achieve their goal in most cases. For Joe, however, it
is likely that booking multiple courts is a rare case and would have to contact
a sports centre for information on duration and multiple court bookings.

Other possible features the application could have include additional details
such as `court surface' and something that would help the user to determine the
quality of the facilities such as a user-ratings system

The findings from the cognitive walkthroughs show that most of the steps needed
to complete a task were possible and the personas would be able to complete the
task. In some cases, the app would need the extra features mentioned above for
a user to fully achieve their goal. It also highlighted features that may be
difficult for the user to understand like the `group by' selection.

%TODO
The prototype was tested by 30 people who were asked to use the prototype
system and then complete a questionnaire (see Appendix~A). The results from the
questionnaire show that … (Also add notes on observations/comments made when
people were using the app – what did they find easiest?/most confusing?)

Overall, the second prototype improves the functionality of the three previous
designs but there are still features that need further development, and
functions, such as error prevention, that should be added.
