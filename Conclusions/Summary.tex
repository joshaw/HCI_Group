\section{Summary}
\label{sec:summary}

Our aim was to design a mobile booking system for sports centres that would
allow users to search multiple facilities at the same time and compare results.
This is something that can be done with many booking applications for other
services.

Through our research into mobile booking apps and sports facilities websites,
we identified key design features for our application. We also found that the
interface may have to present a large amount of data. Readability and filtering
results would have to be considered as well as an easy-to-use, clean interface.
We then used these findings to design our first generation prototypes. These
were based on the three navigation styles outlined by Apple Inc; hierarchical,
flat, and content-driven.

We evaluated the first generation prototypes using Nielsen’s heuristics and
user personas. We found that each design had features that were useful and
satisfied specific scenarios. These were then incorporated into the design of
the second prototype.

The second prototype was able to achieve more than the previous designs but the
cognitive walkthroughs and questionnaire results show there are some areas that
would need further development if this application were to be taken forward.

\subsection{Next Steps}
\label{sub:next_steps}

% TODO

\subsection{Team Analysis}
\label{sub:team_analysis}

Our approach to the project was well structured in that we researched related
existing systems and work that addresses similar principles and techniques,
this helped us better prepare when designing our system. We were also very
cautious in assuming to understand the user which can be a common mistake to
make being the designer of the system. The user personas and scenarios were
created to avoid this problem. We ensured that we used a variety of inspection
and empirical evaluation methods on our final design to expose any issues which
users may face when interacting with the interface.  In terms of changing and
improving our approach we could have spent more time incorporating other
features such as showing the rating and have the user request a notification
nearer to the time of their booking. The lack of such features came up in our
evaluation but were discarded from the final design. We could have increased
the reliability and validity of our SUS results by having more participants
from a more varied and wider demographic.

As a group, we met weekly and worked well in distributing our work fairly and
evenly. Each week we each had our own task and every member of the group worked
hard to meet the deadlines. Our work was kept well organized on Google Drive so
any member of the group could view and make changes on all the documents
created.
