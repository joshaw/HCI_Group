In order to fully test and improve the prototypes that have been designed, a
set of user personas have been developed. Together, they represent a very wide
range of use cases, user abilities, and user preferences.

The four user personas that have been examined specifically are,
\begin{itemize}
	\item and elderly user,
	\item a proffessional working person,
	\item a student at university, and
	\item a younger child.
\end{itemize}

Together, the profiles for these example users include several different
requirements and restrictions, as well as demands on accessibility. They have
been chosen as examples of real world user situations that would require
certain aspects of the design to be carefully considered to maximise
useability.

By using these personas as a limiting guide, emphasis can be placed on ensuring
that all aspects of the final design are suitable for as wide a range of users
as is appropriate, and allows, if nessessary, the tayloring of this application
to a particular subset of the population.

The table below demonstrates the wide range of use cases encompassed by these
personas and allows the relative abilities and requirements to be compared
(E --- elderly, W --- working, S --- student and C --- child).

\begin{table}[htbp]
	\centering
		\begin{tabu} to 0.8\textwidth {l Y Y Y Y Y}
 			\multicolumn{3}{c}{Not Important} &  & \multicolumn{2}{r}{Very Important} \\
 	 		 & 1 & 2 & 3 & 4 & 5 \\
			\midrule
			Time & E & & S & C & W \\
			Location & W & & C & E & S \\
			Sport & W & C & E & & S \\
			Price & W & E & & C & S \\
			\\
			\multicolumn{3}{c}{Incompetent} &  & \multicolumn{2}{r}{Competent} \\
			\midrule
			Technicallity & E & & W & & S/C
		\end{tabu}
\end{table}
\newpage
