\fullwidth%
\section{First Generation Prototype Conclusion}

Following the evaluation of the first prototypes, there are a number
of features we plan to include in our second prototype.

\begin{center}
	\renewcommand{\arraystretch}{2}
	\begin{longtabu}{c p{0.25\linewidth} X}
		\toprule
		\textbf{Prototype} & \textbf{Feature} & \textbf{Comment}\\
		\midrule
		1 & Map & Intuitive to use due to ubiquitous use in most other
		applications.\\
		1 & Sports icons & Removes clutter from the page; important for a small
		screen. \\
		1 & Date selection \& swiping to select & Allows quick and easy input
		of date selection. \\
		1 & Weather indication & Provides relevant further information for user
		scenarios where an outdoor sport was preferred. \\
		1 & Location details & Provides space to inform user on wheelchair
		access, parking information and weather predictions. \\
		\midrule
		2 & Home screen & Clear options help the user quickly identify what
		part of the application they intend to use. \\
		2 & Booking cancellation & Satisfies user scenarios in cancelling
		booking.  Will add a message to inform the user of particular venue
		rules on cancellation. \\
		2 & Wheelchair accessible search & Satisfies user scenarios where
		disabled access to location was required. \\
		2 & Error messages & Helps users understand how to use the applications
		and gives guidance where they've made errors. \\
		2 & Help button & Provides instructions for users \\
		\midrule
		3 & User details & Allows the user to personalise the application to
		their preferences, speeding up their searches. \\
		3 & Current/previous bookings & Aids the user in both cancelling and
		sharing bookings, both common user scenarios. \\
		3 & Share button & Satisfies user scenarios that requires sharing
		information about a booking with friends/relatives. \\
		3 & Search options non-exhaustive & Matches real world situations and
		user scenarios where the user does not have specific criteria for each
		individual search option. \\
		3 & Back buttons & To an extent, allowed the user to undo some errors
		in navigating through the application. \\
		\bottomrule
	\end{longtabu}
\end{center}

The evaluations, particularly those against the user scenarios, highlighted a
number of features which are missing from all of the initial prototypes:

\begin{itemize}
	\item The ability to make several bookings at once.
	\item Confirmation messages when performing certain actions, such as making
		a booking.
	\item Grouping sports, such as team or outdoor sports, when searching for a
		sport.
	\item Clarification on booking cancellations as some venues may not allow
		cancellations at all or at least not without a certain amount of
		notice.
	\item Including public transport options in the location search
	\item An indication of location quality, such as through user reviews.
\end{itemize}

\subsection{Evaluation of Tools}

\subsubsection{Proto.io}

Proto.io\cite{protoio} is a useful tool as it's specifically for creating
mobile prototypes.  It has a library of different devices and each has default
UI components, such as buttons and lists built-in. Once a number of screens
have been designed, they can be linked together. For example, a button could
link to the next screen, which helps to visualise how the application could
work.  However, a free account only lasts 2 weeks, which means that it isn’t
possible to keep a prototype and preview it’s functions after 14 days.

Overall, it has a very easy to use `drag-and-drop' interface, and the gridlines
are helpful in positioning different components.

\subsubsection{Balsamiq}

Based on the simplicity of the prototypes, Balsamiq\cite{balsamiq} was a useful
mock-up tool to use as it allowed the prototype to be completed with a hand
drawn effect.  The selection of pre-drawn widgets were enough to design
everything required of the prototype with the exception of a home page icon.

A major drawback of the tool was the inability to custom design a widget that
was not already available.  This meant there was no way of personalizing our
design.

\subsubsection{Possible Alternatives}

There are a large number of other tools which we could use to develop the
second prototype. These range from applications not specifically designed for
prototyping. There are drawing packages such as Adobe Photoshop, presentation
software such as Microsoft PowerPoint and also software allowing sophisticated
animation and interactivity such as Adobe Flash. These could be used together
or individually to create a prototype.

\subsubsection{Drawing Packages}

The drawing package would allow a very high quality appearance for each screen.
We could draw each screen exactly as we wanted it to look like. However, this
would come at a cost of time and effort, particularly for someone without
experience in the particular drawing package being used. Furthermore, a drawing
package would have to be used in conjunction with other software is we wanted
to include interactivity. Interactivity is vital for our second prototype as it
will allow us to better evaluate the design.

Presentation software Presentation software would facilitate most basic
interaction options between screens but would likely need to be used in
conjunction with a drawing package to a allow higher quality appearance.
However, these interactions are somewhat limited, particularly if we want to
replicate sometimes complicated gestures that are available on modern mobile
phones.

Animation software More sophisticated animation software would allow greater
interactivity options, but at a significant cost as these tools often require a
lot of prior experience to use effectively and quickly. As none of us have much
previous experience with these types of packages, the prototyping would take a
long time. It is important that we create the prototype quickly so we can spend
more time evaluating the design.

\subsubsection{Conclusion}

We will use proto.io as our tool for developing the second prototype. Of all
the options we considered, this tool will allow the highest level of
functionality to be included in the prototype in the least amount of time and
with the least amount of effort. Furthermore, many of the other possible tools
have a learning curve that can increase the time taken to develop the
prototype. The experience gathered from using proto.io in our initial
prototyping will help us avoid this problem. One possible drawback is that this
tool is only free for a limited time through a trial membership.

\restoregeometry%
