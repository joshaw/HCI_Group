% \begin{adjustwidth*}{-4cm}{}
\fullwidth%
\subsection{Conclusion}

Following the evaluation of the first prototypes, there are a number
of features we plan to include in our second prototype.

\begin{center}
	\renewcommand{\arraystretch}{2}
	\begin{longtabu}{c p{0.25\linewidth} X}
		\toprule
		\textbf{Prototype} & \textbf{Feature} & \textbf{Comment}\\
		\midrule
		1 & Map & Intuitive to use due to ubiqituos use in most other
		applications.\\
		1 & Sports icons & Removes clutter from the page; important for a small
		screen. \\
		1 & Date selection \& swiping to select & Allows quick and easy input
		of date selection. \\
		1 & Weather indication & Provides relevant further information for user
		scenarios where an outdoor sport was preferred. \\
		1 & Location details & Provides space to inform user on wheelchair
		access, parking information and weather predictions. \\
		\midrule
		2 & Home screen & Clear options help the user quickly identify what
		part of the app they intend to use. \\
		2 & Booking cancellation & Satisfies user scenarios in cancelling
		booking.  Will add a message to inform the user of particular venue
		rules on cancellation. \\
		2 & Wheelchair accesible search & Satisfies user scenarios where
		disabled access to location was required. \\
		2 & Error messages & Helps users understand how to use the applications
		and gives guidance where they've made errors. \\
		2 & Help button & Provides instructions for users \\
		\midrule
		3 & User details & Allows the user to personalise the app to their
		preferences, speeding up their searches. \\
		3 & Current/previous bookings & Aids the user in both cancelling and
		sharing bookings, both common user scenarios. \\
		3 & Share button & Satisfies user scenarios that requires sharing
		information about a booking with friends/relatives. \\
		3 & Search options non-exhaustive & Matches real world situations and
		user scenarios where the user does not have specific criteria for each
		individual search option. \\
		3 & Back buttons & To an extent, allowed the user to undo some errors
		in navigating through the application. \\
		\bottomrule
	\end{longtabu}
\end{center}
% \end{adjustwidth*}

The evaluations, particularly those against the user scenarios, highlighted a
number of features which are missing from all of the initial prototypes:

\begin{itemize}
	\item The ability to make several bookings at once.
	\item Confirmation messages when performing certain actions, such as making
		a booking.
	\item Grouping sports, such as team or outdoor sports, when searching for a
		sport.
	\item Clarification on booking cancellations as some venues may not allow
		cancellations at all or at least not without a certain amount of
		notice.
	\item Including public transport options in the location search
	\item An indication of location quality, such as through user reviews.
\end{itemize}

\subsection{Evaluation of Tools}

\subsubsection{Proto.io}

To do:

Evalution of proto.io

\subsubsection{Balsamiq}

Based on the simplicity of the prototypes, Balsamiq was a useful mockup tool to
use as it allowed the prototype to be completed with a hand drawn effect. The
selection of pre-drawn widgets were enough to design everything required of the
prototype with the exception of a home page icon.

A major drawback of the tooI was the inability to custom design a widget that
was not already available.
\restoregeometry%
