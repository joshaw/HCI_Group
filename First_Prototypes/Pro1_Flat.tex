\subsection{Prototype 3}
\label{sub:prototype_3}

\subsubsection{Presentation}
\textbf{Tools:} proto.io

\paragraph{Rationale}

This prototype is based on the `flat' design of other booking apps. The search
criteria is spread across a number of pages and results are displayed as a
list.

\paragraph{Home Screen}
\marginpar{%
	\begin{overpic}[width=\marginparwidth]
		{img/firstPrototypes/Pro3Main}
	\end{overpic}
	\captionof{figure}{Home screen is also the search page.}\label{fig:Pro3Home}
}

The home screen is also a search page. Users are able to search by sport,
location, distance, date and time or a combination of these options; all of
these have the default value of `Any' if the user decides not to enter a
specific value or range.

By selecting a search option, such as `sport', the user will be directed to
another page where they can specify a sport or combination of sports using a
checklist interface similar to the previous prototypes. The `date' section
would allow the user to select a specific date, a variety of dates or between
two dates using a calandar interface. The user can select a timeframe. E.g.
after 5pm, before 12pm or between 4pm and 8pm using the `time page'. Distance
can also be selected by range (e.g up to 5 miles) Location can be selected from
a drop-down list of cities, the user can also type their location or use GPS
for their current location.

Once the user has selected their options they can use the `Search' button to
see their results.

\paragraph{Details}
\marginpar{%
	\begin{overpic}[width=\marginparwidth]
		{img/firstPrototypes/Pro3Details}
	\end{overpic}
	\captionof{figure}{Basic information screen.}\label{fig:Pro3Details}
}

The `Details' button in the navigation bar can store information about the user
such as their age, which can help them to find offers that are relevant to them
or discounts can be applied to the price during the search

Basic information about the user can be stored locally to apply discounts and
include relevant offers

\paragraph{Results}
\marginpar{%
	\begin{overpic}[width=\marginparwidth]
		{img/firstPrototypes/Pro3Results_1}
	\end{overpic}
	\captionof{figure}{Results displayed as a list.}\label{fig:Pro3Results_1}
}

The results page allows the user to see their search criteria as well as a list
of available facilities. These can be sorted by price or distance.

The user can go back to change the search criteria using the `Back' button on
the navigation bar or select one of the results in the list for more
information.
\marginpar{%
	\begin{overpic}[width=\marginparwidth]
		{img/firstPrototypes/Pro3Results_2}
	\end{overpic}
	\captionof{figure}{Further information is available.}\label{fig:Pro3Results_2}
}

Once the user chooses an available result, they can see further information on
the facilities selected such as pricing, address, location and contact
information. The user can choose to `share' this information with others or
`book' the facilities using the buttons at the bottom of the screen.

The user can find out more about their current bookings by selecting them from
the main `bookings' page. For previous bookings, the `cancel' button could
change to `book again'.

\paragraph{Tab bar}

There are two tabs on the bar at the bottom of the screen;
\begin{itemize}
	\item `Offers' tab, shows available offers. A user could choose to use
		this to search for facilities by available offers
	\item `Bookings' tab, users can keep track of their current and previous
		bookings
\end{itemize}

\marginpar{%
	\begin{overpic}[width=\marginparwidth]
		{img/firstPrototypes/Pro3Bookings_1}
	\end{overpic}
	\captionof{figure}{The bookings tab}\label{fig:Pro3Bookings_1}
}
\marginpar{%
	\begin{overpic}[width=\marginparwidth]
		{img/firstPrototypes/Pro3Bookings_2}
	\end{overpic}
	\captionof{figure}{Further information is available.}\label{fig:Pro3Bookings_2}
}

\subsubsection{Evaluation}

\begin{center}
	\renewcommand{\arraystretch}{2}
	\begin{longtable}{p{0.3\textwidth} c p{0.5\textwidth}}
		\toprule
		\textbf{Criteria} & \textbf{Rating} & \textbf{Comment}\\
		\midrule
		Visibility of system status & $+$ & There are only two states in this
		application --- the search screen and the results page.\\

		Match between system and the real world & $+$ & Most other booking
		applications have a similar layout of a search page followed by a list
		of results (E.g.\ trainline, redspottedhanky). It should be easy for a
		user who is familiar with this format to use this design.\\

		User control and freedom & $+$ & `Back' button in the navigation bar
		allows the user to change elements of the search criteria.\\

		Consistency and standards & $+$ & Information is displayed in a similar
		way throughout the application. Eg.  Bookings and Results both use
		lists and selecting a particular item in the list leads to a page with
		more specific information.\\

		Error prevention & $-$ & There is no way for a user to tell if they
		have made a mistake or where the errors are. A pop-up notification
		could supply this information when the user presses the `search'
		button.\\

		Recognition rather than recall & $+$ & Search criteria is displayed on
		the main page and in the results section.\\

		Flexibility and efficiency of use & $0$ & Novice users may not find this
		format easy-to-use without instructions.  Experienced users could also
		search for offers, or their current/previous bookings using the tab bar
		in addition to using the home screen. \\

		Aesthetic and minimalist design & $0$ & Keeping the search options on
		different pages prevents the home screen from becoming cluttered.
		However, presenting the results as a list may not be helpful for users
		who do not select a specific sport, date, time or location.\\

		Help users recognize, diagnose, and recover from errors & $-$ & There
		is no way for a user to tell if they have made an error. The only
		option available is to go `back' and change the search criteria.\\

		Help and documentation & $-$ & Currently there are no instructions
		available on how to use the app.\\
		\bottomrule
	\end{longtable}
\end{center}

\begin{center}
	\renewcommand{\arraystretch}{2}
	\begin{longtable}{p{0.12\textwidth} p{0.3\textwidth} c p{0.45\textwidth}}
		\toprule
		\textbf{User} & \textbf{Scenario} & \textbf{Rating} & \textbf{Comment}\\
		\midrule
		\textbf{Elderly} & Searching for new sports in the area and notifying
		his wife of the booking. & $+$ & Howard can search using the location and
		distance criteria for searching for sorts facilities locally. He can
		also send the details of his bookings to his wife by using the `share'
		button.\\

		& Racquet sport with 4 friends on Friday & $0$ & Howard can select the
		individual sports from a list, there is no option at the moment for
		racquet sports. He can also choose a Friday, but wouldn't be able to
		bulk book for a regular session in-app.\\

		& Swimming nearby with knee pain & $-$ & It isn't possible to search
		for facilities that have disabled access, this could be something to
		include in the `details' section and in the information pages of
		individual sports centres.\\

		\midrule \textbf{Working} & Team sport on Friday including screen
		sharing with friends& $-$ & Janet can select individual sports like
		netball, football, etc.\  as there is no option for `team sports' and
		dates. She wouldn't be able to share all the results with her friends
		but could share individual bookings she selects.\\

		& Change/cancel booking at late notice & $0$ & Using the `Bookings'
		tab, Janet could find her booking and cancel it using the `cancel'
		button, or use the information to contact the sports facility to change
		her booking.\\

		& Outdoor sport early on Saturday & $+$ & Currently no quick filters
		for `outdoor' sports, Janet would have to go through the list of all
		possible sports and select those that she knows are outdoors.  Or it
		could be easier for Janet to select Saturday and mornings using the
		date and time sections and see what sports are availabe.\\

		\midrule \textbf{Student} & Tennis court at specific times & $0$ &
		Jenny can select tennis only but may have to search a few times to find
		suitable slots for the different times she is free.\\

		& Weekday evening session must be on clay & $0$ & Jenny can select the
		whole week and hours in the evenings in the `date' and `time' sections.
		She would have to check individual sports facilities to see the types
		of courts available.\\

		& Weekly practice with friend with reminders & $0$ & There isn't a way
		for Jenny to book weekly sessions but could book one session a week and
		share the information with a friend using the `share' button.\\

		\midrule \textbf{Child} & Outdoor sport close to home or on a bus route
		with coach & $0$ & It currently isn't possible to select `outdoor'
		sports but he could choose a variety of sports in the sports section
		and can sort by distance. It wouldn't be possible to know if the
		facilities are close to a bus route but could check with the facilities
		by contacting them.\\

		& Booking several squash courts for after school tournament & $0$ & It
		isn't possible for Joe to book several courts at one time.

		Could have some kind of rating system to the location description on
		the bookings page and some way to search for highly rated locations.\\

		& Looking for high quality tennis court & $0$ & It could be possible to
		include other users ratings of each facility and sort results by these
		ratings.\\
		\bottomrule
	\end{longtable}
\end{center}
