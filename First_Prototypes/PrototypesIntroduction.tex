The design process is aided by the generation and evaluation of a number of
first and second generation prototypes. These will be assessed against several
specific criteria as well as the user personas defined in Section~%
\ref{sec:user_personas}. Using the results from these evaluations, the
best aspects of the first prototypes will be used to inform the second
generation.

When evaluating the initial designs, each of the potential scenarios are
examined and the prototype tested to see if it provides the required or desired
functionality. In addition to theses real world situations, the designs are
tested against a set of heuristics called Neilsen’s heuristics which

\begin{description}

	\item[Visibility of system status] the activity that is currently being
		performed should be clear to the user, and the status of that activity
		should be clear. For example, if a process is running, waiting, or
		completed.

	\item[Match between system and the real world] using standard conventions
		for ordering items makes them easier to search through and select. Also
		the wording of buttons, labels and information should be familiar to
		the user. However, the computer system should not try to immitate a
		physical object directly, i.e.\ skewomorphism.

	\item[User control and freedom] the user should be in control of the
		system. The system should work for them, but provide the ability to
		undo mistaken actions.

	\item[Consistency and standards] any methods for interacting with the
		system should be uniform accross different platforms so that users do
		not need to relearn to use the system.

	\item[Error prevention] reducing the possibility of errors, and the ability
		for the user to provide data that could cause an error is better than
		recovering from errors. If an error does happen, then giving the user
		information is generally better than leaving them without knowing what
		happened.

	\item[Recognition rather than recall] having navigational elements clearly
		visible and reachable means that the user does not need to remember how
		to use the application, instead the instructions are effectively
		onscreen.

	\item[Flexibility and efficiency of use] catering to advanced users without
		distracting or confusing the novice allows the system to be used by a
		wider range of people.

	\item[Aesthetic and minimalist design] including irrelevant data, or
		information that is only needed infrequently can be distracting.
		Reducing the number of visual stimulii presented to the user can
		increase speed and efficiency.

	\item[Help users recognize, diagnose, and recover from errors] easy to
		read, simple error messages, briefly explaining what happened can help
		the user to not get into the same situation again.

	\item[Help and documentation] providing documentation in a well structured
		way can help the tentative user to use the basic functionality and the
		advanced user find more.

\end{description}
