The design process is aided by the generation and evaluation of a number
of first and second generation prototypes. These will be assessed against
several specific criteria as well as the user personas defined in
section~\ref{sec:user_personas}. Using the results from these evaluations, the
best aspects of the first prototypes will be used to inform the second
generation.

When evaluating the initial designs, each of the potential scenarios are
examined and the prototype tested to see if it provides the required or desired
functionality. In addition to theses real world situations, the designs are tested
against a set of heuristics called Neilsen’s heuristics which

\begin{description}

	\item[Visibility of system status] the activity that is currently being
	performed should be clear to the user, and the status of that activity
	should be clear. For example, if a process is running, waiting, or
	completed.

	\item[Match between system and the real world] using standard conventions
	for ordering items makes them easier to search through and select. Also the
	wording of buttons, labels and information should be familiar to the
	user. However, the computer system should not try to immitate a physical
	object directly, i.e.\ skewomorphism.

	\item[User control and freedom] the user should be in control of the
	system. The system should work for them, but provide the ability to undo
	mistaken actions.

	\item[Consistency and standards] any methods for interacting with the
	system should be uniform accross different platforms so that users do not
	need to relearn to use the system.

	\item[Error prevention] reducing the possibility of errors, and the ability
	for the user to provide data that could cause 
	\item[Recognition rather than recall]
	\item[Flexibility and efficiency of use]
	\item[Aesthetic and minimalist design]
	\item[Help users recognize, diagnose, and recover from errors]
	\item[Help and documentation]
\end{description}
