%% !TEX TS-program = pdflatex
%% !TEX encoding = UTF-8 Unicode
%
\documentclass[a4paper,10pt]{article}
\usepackage[utf8]{inputenc}
%
%%%% PAGE DIMENSIONS
%\usepackage[top=0.6in, left=0.8in, right=0.8in, bottom=0.7in]{geometry} % to change the page dimensions
\usepackage[marginpar=5cm,
	marginparsep=1cm,
	top=1in,
	bottom=1.5in,
	left=3cm,
	right=7cm]{geometry}

\pagestyle{empty}
\pagenumbering{gobble}

%%%% PACKAGES
\usepackage[parfill]{parskip}
\usepackage{wasysym}% provides \ocircle and \Box
\usepackage{enumitem}% easy control of topsep and leftmargin for lists
\usepackage{color}% used for background color
\usepackage{forloop}% used for \Qrating and \Qlines
\usepackage{ifthen}% used for \Qitem and \QItem
\usepackage{typearea}
\areaset{17cm}{26cm}
% \setlength{\topmargin}{-1cm}
\usepackage{scrpage2}
\usepackage{graphicx}
\usepackage{setspace}
\usepackage{tabu}

\def\MBox{\Large\ensuremath{\Box}}

\usepackage{amssymb}
\usepackage{array}
\usepackage{tabularx}

\renewcommand{\tabularxcolumn}[1]{m{#1}} % redefine 'X' to use 'm'

%% \Qq = Questionaire question. Oh, this is just too simple. It helps
%% making it easy to globally change the appearance of questions.
\newcommand{\Qq}[1]{\textbf{#1}}

%% \QO = Circle or box to be ticked. Used both by direct call and by
%% \Qrating and \Qlist.
\newcommand{\QO}{$\Box$}% or: $\ocircle$

%% \Qrating = Automatically create a rating scale with NUM steps, like
%% this: 0--0--0--0--0.
\newcounter{qr}
\newcommand{\Qrating}[1]{\QO\forloop{qr}{1}{\value{qr} < #1}{---\QO}}

%% \Qline = Again, this is very simple. It helps setting the line
%% thickness globally. Used both by direct call and by \Qlines.
\newcommand{\Qline}[1]{\rule{#1}{0.6pt}}

%% \Qlines = Insert NUM lines with width=\linewith. You can change the
%% \vskip value to adjust the spacing.
\newcounter{ql}
\newcommand{\Qlines}[1]{\forloop{ql}{0}{\value{ql}<#1}{\vskip0em\Qline{\linewidth}}}

%% \Qlist = This is an environment very similar to itemize but with
%% \QO in front of each list item. Useful for classical multiple
%% choice. Change leftmargin and topsep accourding to your taste.
\newenvironment{Qlist}{%
  \renewcommand{\labelitemi}{\QO}
  \begin{itemize}[leftmargin=1.5em,topsep=-.5em]
}{%
  \end{itemize}
}

%% \Qtab = A "tabulator simulation". The first argument is the
%% distance from the left margin. The second argument is content which
%% is indented within the current row.
\newlength{\qt}
\newcommand{\Qtab}[2]{
  \setlength{\qt}{\linewidth}
  \addtolength{\qt}{-#1}
%  \hfill\parbox[t]{\qt}{\raggedright #2}
\parbox[t]{\qt}{\raggedright #2}

}

%% \Qitem = Item with automatic numbering. The first optional argument
%% can be used to create sub-items like 2a, 2b, 2c, ... The item
%% number is increased if the first argument is omitted or equals 'a'.
%% You will have to adjust this if you prefer a different numbering
%% scheme. Adjust topsep and leftmargin as needed.
\newcounter{itemnummer}
\newcommand{\Qitem}[2][]{% #1 optional, #2 notwendig
  \ifthenelse{\equal{#1}{}}{\stepcounter{itemnummer}}{}
  \ifthenelse{\equal{#1}{a}}{\stepcounter{itemnummer}}{}
  \begin{enumerate}[topsep=2pt,leftmargin=2.8em]
  \item[\textbf{\arabic{itemnummer}#1.}] #2
  \end{enumerate}
}

%% \QItem = Like \Qitem but with alternating background color. This
%% might be error prone as I hard-coded some lengths (-5.25pt and
%% -3pt)! I do not yet understand why I need them.
\newcounter{itemoddeven}
\newlength{\gb}
\newcommand{\QItem}[2][]{% #1 optional, #2 notwendig
  \setlength{\gb}{\linewidth}
  %\setlength{\gb}{19cm}
  \addtolength{\gb}{-5.25pt}

  %\addtolength{\gb}{-5.25pt}
  \ifthenelse{\equal{\value{itemoddeven}}{0}}{%
	\noindent\colorbox{bgeven}{\hskip-3pt\begin{minipage}{\gb}\bigskip\Qitem[#1]{#2}\vspace{5 mm}\end{minipage}}%
	\stepcounter{itemoddeven}%

  }{%
	\noindent\colorbox{bgodd}{\hskip-3pt\begin{minipage}{\gb}\bigskip\Qitem[#1]{#2}\vspace{5 mm}\end{minipage}}%
	\setcounter{itemoddeven}{0}%

  }
}

\newcolumntype{S}{>{\centering\arraybackslash}m{2em}}

%%%%%%%%%%%%%%%%%%%%%%%%%%%%%%%%%%%%%%%%%%%%%%%%%%%%%%%%%%%%

\title{Sports Booking Application \\ System Usability}
\author{%
	Rebecca Devney \\
	Aasima Pathan \\
	Josh Wainwright \\
	Andrew Walker
}
\date{}

\begin{document}

\maketitle

\begin{center}
\begin{tabu}{l X S S S S S}
    &                                                                                                           & \multicolumn{2}{l}{\footnotesize{Disagree}}    &       &   \multicolumn{2}{r}{\footnotesize{Agree}}\\
    &                                                                                                           & 1                                              & 2     & 3     & 4     & 5\\
1.  &\textbf{I think that I would like to use this system frequently}\dotfill                                   & \MBox                                          & \MBox & \MBox & \MBox & \MBox\\
    &                                                                                                           &                                                &       &       &       & \\
2.  &\textbf{I found the system unnecessarily complex}\dotfill                                                  & \MBox                                          & \MBox & \MBox & \MBox & \MBox\\
    &                                                                                                           &                                                &       &       &       & \\
3.  &\textbf{I thought the system was easy to use}\dotfill                                                      & \MBox                                          & \MBox & \MBox & \MBox & \MBox\\
    &                                                                                                           &                                                &       &       &       & \\
4.  &\textbf{I think that I would need the support of a technical person to be able to use this system}\dotfill & \MBox                                          & \MBox & \MBox & \MBox & \MBox\\
    &                                                                                                           &                                                &       &       &       & \\
5.  &\textbf{I found the various functions in this system were well integrated}\dotfill                         & \MBox                                          & \MBox & \MBox & \MBox & \MBox\\
    &                                                                                                           &                                                &       &       &       & \\
6.  &\textbf{I thought there was too much inconsistency in this system}\dotfill                                 & \MBox                                          & \MBox & \MBox & \MBox & \MBox\\
    &                                                                                                           &                                                &       &       &       & \\
7.  &\textbf{I would imagine that most people would learn to use this system very quickly}\dotfill              & \MBox                                          & \MBox & \MBox & \MBox & \MBox\\
    &                                                                                                           &                                                &       &       &       & \\
8.  &\textbf{I found the system very cumbersome to use}\dotfill                                                 & \MBox                                          & \MBox & \MBox & \MBox & \MBox\\
    &                                                                                                           &                                                &       &       &       & \\
9.  &\textbf{I felt very confident using the system}\dotfill                                                    & \MBox                                          & \MBox & \MBox & \MBox & \MBox\\
    &                                                                                                           &                                                &       &       &       & \\
10. &\textbf{I needed to learn a lot of things before I could get going with this system}\dotfill               & \MBox                                          & \MBox & \MBox & \MBox & \MBox\\
\end{tabu}%
\end{center}

\bigskip
\subsection*{What was good?}
\vfill
\subsection*{What could be made better?}
\vfill
\vfill
Gender:

Age:

\end{document}
