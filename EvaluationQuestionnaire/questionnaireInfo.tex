\section{System Usability Scale Questionnaire}
\label{sec:system_usability_scale_questionnaire}

In order to gather information about how real users interact with the system, a
System Usability Scale (SUS) Questionnaire is used. This asks users to give a
mark for each of ten questions relating the their experience with the
application. They are first given the opportunity to use the system, to
interact in any way the like and to play with the functionality that has so far
been implemented. They are then presented the questions. The questionnaire that
was used can be seen in Appendix~A.

This questionnaire was given to 30 people who had first familiarised themselves
with the system. The set of questions are all scored from 1 to 5. In order to
be able to compare them correctly, these are normalised from 1 to 4. The graph
in Figure~\ref{fig:evalGraph} shows the average score for each question.
Highlighted in red are the two questions that received the lowest scores and in
green, the highest.

\begin{figure}[h]
\centering
\begin{tikzpicture}{0pt}{0pt}{529pt}{265pt}
	\clip(0pt,265pt) -- (398.238pt,265pt) -- (398.238pt,65.5047pt) -- (0pt,65.5047pt) -- (0pt,265pt);
\begin{scope}
	\clip(38.3934pt,257.472pt) -- (389.204pt,257.472pt) -- (389.204pt,99.3812pt) -- (38.3934pt,99.3812pt) -- (38.3934pt,257.472pt);
	\color[rgb]{0.627451,0.627451,0.643137}
	\draw[line width=0.4pt, dash pattern=on 0.0096cm off 0.032cm, dash phase=0pt, line join=bevel, line cap=rect](38.3934pt,103.145pt) -- (388.451pt,103.145pt);
	\color[rgb]{0.627451,0.627451,0.643137}
	\draw[line width=0.4pt, dash pattern=on 0.0096cm off 0.032cm, dash phase=0pt, line join=bevel, line cap=rect](38.3934pt,106.909pt) -- (388.451pt,106.909pt);
	\draw[line width=0.4pt, dash pattern=on 0.0096cm off 0.032cm, dash phase=0pt, line join=bevel, line cap=rect](38.3934pt,111.426pt) -- (388.451pt,111.426pt);
	\draw[line width=0.4pt, dash pattern=on 0.0096cm off 0.032cm, dash phase=0pt, line join=bevel, line cap=rect](38.3934pt,115.19pt) -- (388.451pt,115.19pt);
	\draw[line width=0.4pt, dash pattern=on 0.0096cm off 0.032cm, dash phase=0pt, line join=bevel, line cap=rect](38.3934pt,122.718pt) -- (388.451pt,122.718pt);
	\draw[line width=0.4pt, dash pattern=on 0.0096cm off 0.032cm, dash phase=0pt, line join=bevel, line cap=rect](38.3934pt,127.235pt) -- (388.451pt,127.235pt);
	\draw[line width=0.4pt, dash pattern=on 0.0096cm off 0.032cm, dash phase=0pt, line join=bevel, line cap=rect](38.3934pt,130.999pt) -- (388.451pt,130.999pt);
	\draw[line width=0.4pt, dash pattern=on 0.0096cm off 0.032cm, dash phase=0pt, line join=bevel, line cap=rect](38.3934pt,134.763pt) -- (388.451pt,134.763pt);
	\draw[line width=0.4pt, dash pattern=on 0.0096cm off 0.032cm, dash phase=0pt, line join=bevel, line cap=rect](38.3934pt,143.044pt) -- (388.451pt,143.044pt);
	\draw[line width=0.4pt, dash pattern=on 0.0096cm off 0.032cm, dash phase=0pt, line join=bevel, line cap=rect](38.3934pt,146.808pt) -- (388.451pt,146.808pt);
	\draw[line width=0.4pt, dash pattern=on 0.0096cm off 0.032cm, dash phase=0pt, line join=bevel, line cap=rect](38.3934pt,150.572pt) -- (388.451pt,150.572pt);
	\draw[line width=0.4pt, dash pattern=on 0.0096cm off 0.032cm, dash phase=0pt, line join=bevel, line cap=rect](38.3934pt,154.337pt) -- (388.451pt,154.337pt);
	\draw[line width=0.4pt, dash pattern=on 0.0096cm off 0.032cm, dash phase=0pt, line join=bevel, line cap=rect](38.3934pt,162.618pt) -- (388.451pt,162.618pt);
	\draw[line width=0.4pt, dash pattern=on 0.0096cm off 0.032cm, dash phase=0pt, line join=bevel, line cap=rect](38.3934pt,166.382pt) -- (388.451pt,166.382pt);
	\draw[line width=0.4pt, dash pattern=on 0.0096cm off 0.032cm, dash phase=0pt, line join=bevel, line cap=rect](38.3934pt,170.898pt) -- (388.451pt,170.898pt);
	\draw[line width=0.4pt, dash pattern=on 0.0096cm off 0.032cm, dash phase=0pt, line join=bevel, line cap=rect](38.3934pt,174.662pt) -- (388.451pt,174.662pt);
	\draw[line width=0.4pt, dash pattern=on 0.0096cm off 0.032cm, dash phase=0pt, line join=bevel, line cap=rect](38.3934pt,182.191pt) -- (388.451pt,182.191pt);
	\draw[line width=0.4pt, dash pattern=on 0.0096cm off 0.032cm, dash phase=0pt, line join=bevel, line cap=rect](38.3934pt,186.707pt) -- (388.451pt,186.707pt);
	\draw[line width=0.4pt, dash pattern=on 0.0096cm off 0.032cm, dash phase=0pt, line join=bevel, line cap=rect](38.3934pt,190.472pt) -- (388.451pt,190.472pt);
	\draw[line width=0.4pt, dash pattern=on 0.0096cm off 0.032cm, dash phase=0pt, line join=bevel, line cap=rect](38.3934pt,194.236pt) -- (388.451pt,194.236pt);
	\draw[line width=0.4pt, dash pattern=on 0.0096cm off 0.032cm, dash phase=0pt, line join=bevel, line cap=rect](38.3934pt,201.764pt) -- (388.451pt,201.764pt);
	\draw[line width=0.4pt, dash pattern=on 0.0096cm off 0.032cm, dash phase=0pt, line join=bevel, line cap=rect](38.3934pt,206.281pt) -- (388.451pt,206.281pt);
	\draw[line width=0.4pt, dash pattern=on 0.0096cm off 0.032cm, dash phase=0pt, line join=bevel, line cap=rect](38.3934pt,210.045pt) -- (388.451pt,210.045pt);
	\draw[line width=0.4pt, dash pattern=on 0.0096cm off 0.032cm, dash phase=0pt, line join=bevel, line cap=rect](38.3934pt,213.809pt) -- (388.451pt,213.809pt);
	\draw[line width=0.4pt, dash pattern=on 0.0096cm off 0.032cm, dash phase=0pt, line join=bevel, line cap=rect](38.3934pt,222.09pt) -- (388.451pt,222.09pt);
	\draw[line width=0.4pt, dash pattern=on 0.0096cm off 0.032cm, dash phase=0pt, line join=bevel, line cap=rect](38.3934pt,225.854pt) -- (388.451pt,225.854pt);
	\draw[line width=0.4pt, dash pattern=on 0.0096cm off 0.032cm, dash phase=0pt, line join=bevel, line cap=rect](38.3934pt,229.618pt) -- (388.451pt,229.618pt);
	\draw[line width=0.4pt, dash pattern=on 0.0096cm off 0.032cm, dash phase=0pt, line join=bevel, line cap=rect](38.3934pt,234.135pt) -- (388.451pt,234.135pt);
	\draw[line width=0.4pt, dash pattern=on 0.0096cm off 0.032cm, dash phase=0pt, line join=bevel, line cap=rect](38.3934pt,241.663pt) -- (388.451pt,241.663pt);
	\draw[line width=0.4pt, dash pattern=on 0.0096cm off 0.032cm, dash phase=0pt, line join=bevel, line cap=rect](38.3934pt,245.427pt) -- (388.451pt,245.427pt);
	\draw[line width=0.4pt, dash pattern=on 0.0096cm off 0.032cm, dash phase=0pt, line join=bevel, line cap=rect](38.3934pt,249.191pt) -- (388.451pt,249.191pt);
	\draw[line width=0.4pt, dash pattern=on 0.0096cm off 0.032cm, dash phase=0pt, line join=bevel, line cap=rect](38.3934pt,253.708pt) -- (388.451pt,253.708pt);
	\color[rgb]{0,0,1}
	\draw[line width=0.5pt, line join=bevel, line cap=rect](38.3934pt,118.954pt) -- (388.451pt,118.954pt);
	\draw[line width=0.5pt, line join=bevel, line cap=rect](38.3934pt,138.528pt) -- (388.451pt,138.528pt);
	\draw[line width=0.5pt, line join=bevel, line cap=rect](38.3934pt,158.853pt) -- (388.451pt,158.853pt);
	\draw[line width=0.5pt, line join=bevel, line cap=rect](38.3934pt,178.427pt) -- (388.451pt,178.427pt);
	\draw[line width=0.5pt, line join=bevel, line cap=rect](38.3934pt,198pt) -- (388.451pt,198pt);
	\draw[line width=0.5pt, line join=bevel, line cap=rect](38.3934pt,217.573pt) -- (388.451pt,217.573pt);
	\draw[line width=0.5pt, line join=bevel, line cap=rect](38.3934pt,237.899pt) -- (388.451pt,237.899pt);
	\color[rgb]{1,1,1}
	\fill(41.9015pt,240.477pt) -- (69.9664pt,240.477pt) -- (69.9664pt,99.3812pt) -- (41.9015pt,99.3812pt) -- (41.9015pt,240.477pt);
	\color[rgb]{0,0,0}
	\draw[line width=0.5pt, line join=miter, line cap=rect](41.9015pt,240.477pt) -- (69.9664pt,240.477pt) -- (69.9664pt,99.3812pt) -- (41.9015pt,99.3812pt) -- (41.9015pt,240.477pt);
	\color[rgb]{1,1,1}
	\fill(76.9826pt,220.716pt) -- (105.047pt,220.716pt) -- (105.047pt,99.3812pt) -- (76.9826pt,99.3812pt) -- (76.9826pt,220.716pt);
	\color[rgb]{0,0,0}
	\draw[line width=0.5pt, line join=miter, line cap=rect](76.9826pt,220.716pt) -- (105.047pt,220.716pt) -- (105.047pt,99.3812pt) -- (76.9826pt,99.3812pt) -- (76.9826pt,220.716pt);
	\color[rgb]{1,1,1}
	\fill(112.064pt,228.62pt) -- (140.129pt,228.62pt) -- (140.129pt,99.3812pt) -- (112.064pt,99.3812pt) -- (112.064pt,228.62pt);
	\color[rgb]{0,0,0}
	\draw[line width=0.5pt, line join=miter, line cap=rect](112.064pt,228.62pt) -- (140.129pt,228.62pt) -- (140.129pt,99.3812pt) -- (112.064pt,99.3812pt) -- (112.064pt,228.62pt);
	\color[rgb]{1,1,1}
	\fill(147.145pt,244.429pt) -- (175.21pt,244.429pt) -- (175.21pt,99.3812pt) -- (147.145pt,99.3812pt) -- (147.145pt,244.429pt);
	\color[rgb]{0,0,0}
	\draw[line width=0.5pt, line join=miter, line cap=rect](147.145pt,244.429pt) -- (175.21pt,244.429pt) -- (175.21pt,99.3812pt) -- (147.145pt,99.3812pt) -- (147.145pt,244.429pt);
	\color[rgb]{1,1,1}
	\fill(182.226pt,229.806pt) -- (210.291pt,229.806pt) -- (210.291pt,99.3812pt) -- (182.226pt,99.3812pt) -- (182.226pt,229.806pt);
	\color[rgb]{0,0,0}
	\draw[line width=0.5pt, line join=miter, line cap=rect](182.226pt,229.806pt) -- (210.291pt,229.806pt) -- (210.291pt,99.3812pt) -- (182.226pt,99.3812pt) -- (182.226pt,229.806pt);
	\color[rgb]{1,1,1}
	\fill(217.307pt,236.525pt) -- (245.372pt,236.525pt) -- (245.372pt,99.3812pt) -- (217.307pt,99.3812pt) -- (217.307pt,236.525pt);
	\color[rgb]{0,0,0}
	\draw[line width=0.5pt, line join=miter, line cap=rect](217.307pt,236.525pt) -- (245.372pt,236.525pt) -- (245.372pt,99.3812pt) -- (217.307pt,99.3812pt) -- (217.307pt,236.525pt);
	\color[rgb]{1,1,1}
	\fill(252.388pt,223.087pt) -- (280.453pt,223.087pt) -- (280.453pt,99.3812pt) -- (252.388pt,99.3812pt) -- (252.388pt,223.087pt);
	\color[rgb]{0,0,0}
	\draw[line width=0.5pt, line join=miter, line cap=rect](252.388pt,223.087pt) -- (280.453pt,223.087pt) -- (280.453pt,99.3812pt) -- (252.388pt,99.3812pt) -- (252.388pt,223.087pt);
	\color[rgb]{1,1,1}
	\fill(287.469pt,234.944pt) -- (315.534pt,234.944pt) -- (315.534pt,99.3812pt) -- (287.469pt,99.3812pt) -- (287.469pt,234.944pt);
	\color[rgb]{0,0,0}
	\draw[line width=0.5pt, line join=miter, line cap=rect](287.469pt,234.944pt) -- (315.534pt,234.944pt) -- (315.534pt,99.3812pt) -- (287.469pt,99.3812pt) -- (287.469pt,234.944pt);
	\color[rgb]{1,1,1}
	\fill(322.55pt,233.758pt) -- (350.615pt,233.758pt) -- (350.615pt,99.3812pt) -- (322.55pt,99.3812pt) -- (322.55pt,233.758pt);
	\color[rgb]{0,0,0}
	\draw[line width=0.5pt, line join=miter, line cap=rect](322.55pt,233.758pt) -- (350.615pt,233.758pt) -- (350.615pt,99.3812pt) -- (322.55pt,99.3812pt) -- (322.55pt,233.758pt);
	\color[rgb]{1,1,1}
	\fill(357.631pt,241.663pt) -- (385.696pt,241.663pt) -- (385.696pt,99.3812pt) -- (357.631pt,99.3812pt) -- (357.631pt,241.663pt);
	\color[rgb]{0,0,0}
	\draw[line width=0.5pt, line join=miter, line cap=rect](357.631pt,241.663pt) -- (385.696pt,241.663pt) -- (385.696pt,99.3812pt) -- (357.631pt,99.3812pt) -- (357.631pt,241.663pt);
	\definecolor{c}{rgb}{0,0,0}
	\fill [pattern color=c, pattern=north west lines](41.9015pt,240.477pt) -- (69.9664pt,240.477pt) -- (69.9664pt,99.3812pt) -- (41.9015pt,99.3812pt) -- (41.9015pt,240.477pt);
	\draw[line width=0.5pt, line join=miter, line cap=rect](41.9015pt,240.477pt) -- (69.9664pt,240.477pt) -- (69.9664pt,99.3812pt) -- (41.9015pt,99.3812pt) -- (41.9015pt,240.477pt);
	\fill [pattern color=c, pattern=north west lines](76.9826pt,220.716pt) -- (105.047pt,220.716pt) -- (105.047pt,99.3812pt) -- (76.9826pt,99.3812pt) -- (76.9826pt,220.716pt);
	\draw[line width=0.5pt, line join=miter, line cap=rect](76.9826pt,220.716pt) -- (105.047pt,220.716pt) -- (105.047pt,99.3812pt) -- (76.9826pt,99.3812pt) -- (76.9826pt,220.716pt);
	\fill [pattern color=c, pattern=north west lines](112.064pt,228.62pt) -- (140.129pt,228.62pt) -- (140.129pt,99.3812pt) -- (112.064pt,99.3812pt) -- (112.064pt,228.62pt);
	\draw[line width=0.5pt, line join=miter, line cap=rect](112.064pt,228.62pt) -- (140.129pt,228.62pt) -- (140.129pt,99.3812pt) -- (112.064pt,99.3812pt) -- (112.064pt,228.62pt);
	\fill [pattern color=c, pattern=north west lines](147.145pt,244.429pt) -- (175.21pt,244.429pt) -- (175.21pt,99.3812pt) -- (147.145pt,99.3812pt) -- (147.145pt,244.429pt);
	\draw[line width=0.5pt, line join=miter, line cap=rect](147.145pt,244.429pt) -- (175.21pt,244.429pt) -- (175.21pt,99.3812pt) -- (147.145pt,99.3812pt) -- (147.145pt,244.429pt);
	\fill [pattern color=c, pattern=north west lines](182.226pt,229.806pt) -- (210.291pt,229.806pt) -- (210.291pt,99.3812pt) -- (182.226pt,99.3812pt) -- (182.226pt,229.806pt);
	\draw[line width=0.5pt, line join=miter, line cap=rect](182.226pt,229.806pt) -- (210.291pt,229.806pt) -- (210.291pt,99.3812pt) -- (182.226pt,99.3812pt) -- (182.226pt,229.806pt);
	\fill [pattern color=c, pattern=north west lines](217.307pt,236.525pt) -- (245.372pt,236.525pt) -- (245.372pt,99.3812pt) -- (217.307pt,99.3812pt) -- (217.307pt,236.525pt);
	\draw[line width=0.5pt, line join=miter, line cap=rect](217.307pt,236.525pt) -- (245.372pt,236.525pt) -- (245.372pt,99.3812pt) -- (217.307pt,99.3812pt) -- (217.307pt,236.525pt);
	\fill [pattern color=c, pattern=north west lines](252.388pt,223.087pt) -- (280.453pt,223.087pt) -- (280.453pt,99.3812pt) -- (252.388pt,99.3812pt) -- (252.388pt,223.087pt);
	\draw[line width=0.5pt, line join=miter, line cap=rect](252.388pt,223.087pt) -- (280.453pt,223.087pt) -- (280.453pt,99.3812pt) -- (252.388pt,99.3812pt) -- (252.388pt,223.087pt);
	\fill [pattern color=c, pattern=north west lines](287.469pt,234.944pt) -- (315.534pt,234.944pt) -- (315.534pt,99.3812pt) -- (287.469pt,99.3812pt) -- (287.469pt,234.944pt);
	\draw[line width=0.5pt, line join=miter, line cap=rect](287.469pt,234.944pt) -- (315.534pt,234.944pt) -- (315.534pt,99.3812pt) -- (287.469pt,99.3812pt) -- (287.469pt,234.944pt);
	\fill [pattern color=c, pattern=north west lines](322.55pt,233.758pt) -- (350.615pt,233.758pt) -- (350.615pt,99.3812pt) -- (322.55pt,99.3812pt) -- (322.55pt,233.758pt);
	\draw[line width=0.5pt, line join=miter, line cap=rect](322.55pt,233.758pt) -- (350.615pt,233.758pt) -- (350.615pt,99.3812pt) -- (322.55pt,99.3812pt) -- (322.55pt,233.758pt);
	\fill [pattern color=c, pattern=north west lines](357.631pt,241.663pt) -- (385.696pt,241.663pt) -- (385.696pt,99.3812pt) -- (357.631pt,99.3812pt) -- (357.631pt,241.663pt);
	\draw[line width=0.5pt, line join=miter, line cap=rect](357.631pt,241.663pt) -- (385.696pt,241.663pt) -- (385.696pt,99.3812pt) -- (357.631pt,99.3812pt) -- (357.631pt,241.663pt);
	\color[rgb]{0.666667,0,0}
	\fill[opacity=0.549996](41.9015pt,99.3812pt) -- (69.9664pt,99.3812pt) -- (69.9664pt,99.3812pt) -- (41.9015pt,99.3812pt) -- (41.9015pt,99.3812pt);
	\color[rgb]{0,0,0}
	\draw[line width=0.5pt, line join=miter, line cap=rect](41.9015pt,99.3812pt) -- (69.9664pt,99.3812pt) -- (69.9664pt,99.3812pt) -- (41.9015pt,99.3812pt) -- (41.9015pt,99.3812pt);
	\color[rgb]{0.666667,0,0}
	\fill[opacity=0.549996](76.9826pt,220.716pt) -- (105.047pt,220.716pt) -- (105.047pt,99.3812pt) -- (76.9826pt,99.3812pt) -- (76.9826pt,220.716pt);
	\color[rgb]{0,0,0}
	\draw[line width=0.5pt, line join=miter, line cap=rect](76.9826pt,220.716pt) -- (105.047pt,220.716pt) -- (105.047pt,99.3812pt) -- (76.9826pt,99.3812pt) -- (76.9826pt,220.716pt);
	\color[rgb]{0.666667,0,0}
	\fill[opacity=0.549996](252.388pt,223.087pt) -- (280.453pt,223.087pt) -- (280.453pt,99.3812pt) -- (252.388pt,99.3812pt) -- (252.388pt,223.087pt);
	\color[rgb]{0,0,0}
	\draw[line width=0.5pt, line join=miter, line cap=rect](252.388pt,223.087pt) -- (280.453pt,223.087pt) -- (280.453pt,99.3812pt) -- (252.388pt,99.3812pt) -- (252.388pt,223.087pt);
	\color[rgb]{0.666667,0,0}
	\fill[opacity=0.549996](357.631pt,99.3812pt) -- (385.696pt,99.3812pt) -- (385.696pt,99.3812pt) -- (357.631pt,99.3812pt) -- (357.631pt,99.3812pt);
	\color[rgb]{0,0,0}
	\draw[line width=0.5pt, line join=miter, line cap=rect](357.631pt,99.3812pt) -- (385.696pt,99.3812pt) -- (385.696pt,99.3812pt) -- (357.631pt,99.3812pt) -- (357.631pt,99.3812pt);
	\color[rgb]{0,0.666667,0}
	\fill[opacity=0.450004](41.9015pt,99.3812pt) -- (69.9664pt,99.3812pt) -- (69.9664pt,99.3812pt) -- (41.9015pt,99.3812pt) -- (41.9015pt,99.3812pt);
	\color[rgb]{0,0,0}
	\draw[line width=0.5pt, line join=miter, line cap=rect](41.9015pt,99.3812pt) -- (69.9664pt,99.3812pt) -- (69.9664pt,99.3812pt) -- (41.9015pt,99.3812pt) -- (41.9015pt,99.3812pt);
	\color[rgb]{0,0.666667,0}
	\fill[opacity=0.450004](76.9826pt,99.3812pt) -- (105.047pt,99.3812pt) -- (105.047pt,99.3812pt) -- (76.9826pt,99.3812pt) -- (76.9826pt,99.3812pt);
	\color[rgb]{0,0,0}
	\draw[line width=0.5pt, line join=miter, line cap=rect](76.9826pt,99.3812pt) -- (105.047pt,99.3812pt) -- (105.047pt,99.3812pt) -- (76.9826pt,99.3812pt) -- (76.9826pt,99.3812pt);
	\color[rgb]{0,0.666667,0}
	\fill[opacity=0.450004](147.145pt,244.429pt) -- (175.21pt,244.429pt) -- (175.21pt,99.3812pt) -- (147.145pt,99.3812pt) -- (147.145pt,244.429pt);
	\color[rgb]{0,0,0}
	\draw[line width=0.5pt, line join=miter, line cap=rect](147.145pt,244.429pt) -- (175.21pt,244.429pt) -- (175.21pt,99.3812pt) -- (147.145pt,99.3812pt) -- (147.145pt,244.429pt);
	\color[rgb]{0,0.666667,0}
	\fill[opacity=0.450004](357.631pt,241.663pt) -- (385.696pt,241.663pt) -- (385.696pt,99.3812pt) -- (357.631pt,99.3812pt) -- (357.631pt,241.663pt);
	\color[rgb]{0,0,0}
	\draw[line width=0.5pt, line join=miter, line cap=rect](357.631pt,241.663pt) -- (385.696pt,241.663pt) -- (385.696pt,99.3812pt) -- (357.631pt,99.3812pt) -- (357.631pt,241.663pt);
\end{scope}
\begin{scope}
	\color[rgb]{0,0,0}
	\pgftext[center, base, at={\pgfpoint{6.77531pt}{178.803pt}},rotate=90]{\selectfont{\textbf{Average Score}}}
	\color[rgb]{0,0,0}
	\pgftext[center, base, at={\pgfpoint{22.5844pt}{97.8756pt}}]{\selectfont{0}}
	\pgftext[center, base, at={\pgfpoint{20.3259pt}{117.449pt}}]{\selectfont{0.5}}
	\pgftext[center, base, at={\pgfpoint{22.5844pt}{137.022pt}}]{\selectfont{1}}
	\pgftext[center, base, at={\pgfpoint{20.3259pt}{157.348pt}}]{\selectfont{1.5}}
	\pgftext[center, base, at={\pgfpoint{21.8316pt}{176.921pt}}]{\selectfont{2}}
	\pgftext[center, base, at={\pgfpoint{20.3259pt}{196.494pt}}]{\selectfont{2.5}}
	\pgftext[center, base, at={\pgfpoint{22.5844pt}{216.067pt}}]{\selectfont{3}}
	\pgftext[center, base, at={\pgfpoint{20.3259pt}{236.393pt}}]{\selectfont{3.5}}
	\pgftext[center, base, at={\pgfpoint{21.8316pt}{255.966pt}}]{\selectfont{4}}
	\draw[line width=0.5pt, line join=bevel, line cap=rect](38.3934pt,103.145pt) -- (34.6294pt,103.145pt);
	\draw[line width=0.5pt, line join=bevel, line cap=rect](38.3934pt,106.909pt) -- (34.6294pt,106.909pt);
	\draw[line width=0.5pt, line join=bevel, line cap=rect](38.3934pt,111.426pt) -- (34.6294pt,111.426pt);
	\draw[line width=0.5pt, line join=bevel, line cap=rect](38.3934pt,115.19pt) -- (34.6294pt,115.19pt);
	\draw[line width=0.5pt, line join=bevel, line cap=rect](38.3934pt,122.718pt) -- (34.6294pt,122.718pt);
	\draw[line width=0.5pt, line join=bevel, line cap=rect](38.3934pt,127.235pt) -- (34.6294pt,127.235pt);
	\draw[line width=0.5pt, line join=bevel, line cap=rect](38.3934pt,130.999pt) -- (34.6294pt,130.999pt);
	\draw[line width=0.5pt, line join=bevel, line cap=rect](38.3934pt,134.763pt) -- (34.6294pt,134.763pt);
	\draw[line width=0.5pt, line join=bevel, line cap=rect](38.3934pt,143.044pt) -- (34.6294pt,143.044pt);
	\draw[line width=0.5pt, line join=bevel, line cap=rect](38.3934pt,146.808pt) -- (34.6294pt,146.808pt);
	\draw[line width=0.5pt, line join=bevel, line cap=rect](38.3934pt,150.572pt) -- (34.6294pt,150.572pt);
	\draw[line width=0.5pt, line join=bevel, line cap=rect](38.3934pt,154.337pt) -- (34.6294pt,154.337pt);
	\draw[line width=0.5pt, line join=bevel, line cap=rect](38.3934pt,162.618pt) -- (34.6294pt,162.618pt);
	\draw[line width=0.5pt, line join=bevel, line cap=rect](38.3934pt,166.382pt) -- (34.6294pt,166.382pt);
	\draw[line width=0.5pt, line join=bevel, line cap=rect](38.3934pt,170.898pt) -- (34.6294pt,170.898pt);
	\draw[line width=0.5pt, line join=bevel, line cap=rect](38.3934pt,174.662pt) -- (34.6294pt,174.662pt);
	\draw[line width=0.5pt, line join=bevel, line cap=rect](38.3934pt,182.191pt) -- (34.6294pt,182.191pt);
	\draw[line width=0.5pt, line join=bevel, line cap=rect](38.3934pt,186.707pt) -- (34.6294pt,186.707pt);
	\draw[line width=0.5pt, line join=bevel, line cap=rect](38.3934pt,190.472pt) -- (34.6294pt,190.472pt);
	\draw[line width=0.5pt, line join=bevel, line cap=rect](38.3934pt,194.236pt) -- (34.6294pt,194.236pt);
	\draw[line width=0.5pt, line join=bevel, line cap=rect](38.3934pt,201.764pt) -- (34.6294pt,201.764pt);
	\draw[line width=0.5pt, line join=bevel, line cap=rect](38.3934pt,206.281pt) -- (34.6294pt,206.281pt);
	\draw[line width=0.5pt, line join=bevel, line cap=rect](38.3934pt,210.045pt) -- (34.6294pt,210.045pt);
	\draw[line width=0.5pt, line join=bevel, line cap=rect](38.3934pt,213.809pt) -- (34.6294pt,213.809pt);
	\draw[line width=0.5pt, line join=bevel, line cap=rect](38.3934pt,222.09pt) -- (34.6294pt,222.09pt);
	\draw[line width=0.5pt, line join=bevel, line cap=rect](38.3934pt,225.854pt) -- (34.6294pt,225.854pt);
	\draw[line width=0.5pt, line join=bevel, line cap=rect](38.3934pt,229.618pt) -- (34.6294pt,229.618pt);
	\draw[line width=0.5pt, line join=bevel, line cap=rect](38.3934pt,234.135pt) -- (34.6294pt,234.135pt);
	\draw[line width=0.5pt, line join=bevel, line cap=rect](38.3934pt,241.663pt) -- (34.6294pt,241.663pt);
	\draw[line width=0.5pt, line join=bevel, line cap=rect](38.3934pt,245.427pt) -- (34.6294pt,245.427pt);
	\draw[line width=0.5pt, line join=bevel, line cap=rect](38.3934pt,249.191pt) -- (34.6294pt,249.191pt);
	\draw[line width=0.5pt, line join=bevel, line cap=rect](38.3934pt,253.708pt) -- (34.6294pt,253.708pt);
	\draw[line width=0.5pt, line join=bevel, line cap=rect](38.3934pt,99.3812pt) -- (31.6181pt,99.3812pt);
	\draw[line width=0.5pt, line join=bevel, line cap=rect](38.3934pt,118.954pt) -- (31.6181pt,118.954pt);
	\draw[line width=0.5pt, line join=bevel, line cap=rect](38.3934pt,138.528pt) -- (31.6181pt,138.528pt);
	\draw[line width=0.5pt, line join=bevel, line cap=rect](38.3934pt,158.853pt) -- (31.6181pt,158.853pt);
	\draw[line width=0.5pt, line join=bevel, line cap=rect](38.3934pt,178.427pt) -- (31.6181pt,178.427pt);
	\draw[line width=0.5pt, line join=bevel, line cap=rect](38.3934pt,198pt) -- (31.6181pt,198pt);
	\draw[line width=0.5pt, line join=bevel, line cap=rect](38.3934pt,217.573pt) -- (31.6181pt,217.573pt);
	\draw[line width=0.5pt, line join=bevel, line cap=rect](38.3934pt,237.899pt) -- (31.6181pt,237.899pt);
	\draw[line width=0.5pt, line join=bevel, line cap=rect](38.3934pt,257.472pt) -- (31.6181pt,257.472pt);
	\draw[line width=0.5pt, line join=bevel, line cap=rect](38.3934pt,257.472pt) -- (38.3934pt,99.3812pt);
	\pgftext[center, base, at={\pgfpoint{215.681pt}{67.7631pt}}]{\selectfont{\textbf{Question Number}}}
	\pgftext[center, base, at={\pgfpoint{91.4667pt}{82.8194pt}}]{\selectfont{2}}
	\pgftext[center, base, at={\pgfpoint{161.478pt}{82.8194pt}}]{\selectfont{4}}
	\pgftext[center, base, at={\pgfpoint{231.113pt}{82.8194pt}}]{\selectfont{6}}
	\pgftext[center, base, at={\pgfpoint{301.878pt}{82.8194pt}}]{\selectfont{8}}
	\pgftext[center, base, at={\pgfpoint{371.889pt}{82.8194pt}}]{\selectfont{10}}
	\draw[line width=0.5pt, line join=bevel, line cap=rect](91.0903pt,99.3812pt) -- (91.0903pt,92.6059pt);
	\draw[line width=0.5pt, line join=bevel, line cap=rect](161.102pt,99.3812pt) -- (161.102pt,92.6059pt);
	\draw[line width=0.5pt, line join=bevel, line cap=rect](231.113pt,99.3812pt) -- (231.113pt,92.6059pt);
	\draw[line width=0.5pt, line join=bevel, line cap=rect](301.878pt,99.3812pt) -- (301.878pt,92.6059pt);
	\draw[line width=0.5pt, line join=bevel, line cap=rect](371.889pt,99.3812pt) -- (371.889pt,92.6059pt);
	\draw[line width=0.5pt, line join=bevel, line cap=rect](38.3934pt,99.3812pt) -- (389.204pt,99.3812pt);
\end{scope}
\end{tikzpicture}

\caption{}\label{fig:evalGraph}
\end{figure}

\subsection{Summary of Comments}
\label{sub:summary_of_comments}

As well as the ten questions, each user was asked to comment on the good
aspects of the design, and on parts of the implementation that could be
improved.

The main points from the positive comments included the clean-ness of the
displays and screens, the small number of clicks needed to get to any desired
page and the good use of colours and innovative selection methods.

From the `improvements' comments, the sorting and grouping on the results
screen sometimes cause confusion, the repeated bookings functionality on the
booking page was not clear as to its exact purpose for those not looking to
make repeat bookings and, for new users, the AM/PM and weekly selection methods
were not clear.
